\chapter{Introduction}
In this chapter, we will introduce the bachelor thesis and explain it's relevance. We will introduce the concept of Activity Based Computing (ABC) and provide enough background information for the uninformed reader to understand it's purpose. At the same time, we will state an exact problem definition, explain how we approached it and briefly summarise the results. An extensive explanation of Activity Based Computing and the results of the thesis is deferred to subsequent chapters.

\section{Context and Motivation}
Today, work practices are often complex entities with respect to the number of resources and work flows involved. Take for instance a university student who follows several courses at a time. Each course might include literature to read, lectures to attend and exercises to solve. In order for a student to get through an average day, he would need to keep track of his schedule to know which course is taught that day and he would have to solve the exercises which requires resources like the document that defines them, the course web page etc. Furthermore, if he has several lectures the same day or the lecture and the exercises are held in different rooms then he would need to keep track of this and prepare a different set of documents for each activity.

As argued by Bardram \citep{bardram2011} moving away from the typical office desk environment has a negative impact on the challenges that arise from management of parallel activities. Users are left with their own organizing skills for tools when trying to prepare and share their resources before the activity is actually happening. In modern times, computers in various shapes have been utilized to meet and optimize those challenges but they all still provide support on the lowest level only, i.e. allowing the user to browse the file system, open and close applications, manually distribute their documents to collaborators by e-mail etc. 

Earlier experiments have tried to accommodate these challenges using ABC (explained below) and hand held devices like PDAs and tablet computers \citep{bardram2009} running a mobile version of Microsoft Windows. However, they concluded among other things that the screen size of a PDA was too small whereas tablets were too big and heavy which made them less useful at the time being. One particular problem was how to translate a configuration of windows from a desktop environment with a large screen into a mobile environment and still make the windows fit the screen boundaries. 

Since these experiments were carried out mobile devices have become lighter and smaller. Furthermore, specialized mobile operating systems facilitating apps with highly adaptable user interfaces have gained a significant footing thus directly addressing the above mentioned points of critique. Given the challenges that haven been proven to exist in non-office like environments \citep{bardram2011} and the last couple of years' evolution of mobile devices, we wish to investigate how a modern tablet computer with a dedicated operating system like the Apple iPad and iOS can aid university students in their everyday life as an ABC enabled device.

\section{Background}
Activity Based Computing is a computing paradigm originally developed for use with hospitals. It aims at aiding users in managing parallel activities by introducing the higher level notion of an "Activity" that facilitates easy distribution, suspension and resumption and replaces the lower level concept of a file or application as the basic computational unit.

The ABC project started in 2002 and took it's outset in pervasive computing designed for mobile, collaborative and time critical work for clinicians in hospitals. Since 2002 a lot of experiments have been carried out with different families of devices as well as technical implementations at different software levels. The current implementation is a peer-to-peer system that is based on the AEXO infrastructure and targeted for non-PC devices.

Devices that have been used in ABC includes desktops, laptops, tablets and large wall screen displays. Tablets have received a lot of criticism due to their size and weight which made them difficult to handle in front of patients. In one article \citep{bardram2009} it is even explained how nurses were forced to drop the device in the patient's lap in order to interact properly with it.

Despite the fact that ABC was designed for use in hospitals it is introduced as a paradigm for ubiquitous computing that has fostered 6 general principles which can be applied in any context (lately, experiments with ABC has been carried out within the field of biology). These principles are listed and explained in chapter 2.

\section{Goals}
One problem in the everyday life of a student is how to manage all the documents that are associated with a particular lecture. Another problem is how to find out which documents are needed at a particular time of the day and turn instant observations into notes like black board contents, practical experiments etc. Within these problem domains we define 3 goals.

\begin{itemize}
  \item Context awareness
  \item Task management
  \item Mobility
\end{itemize}

In other words, we will investigate how the iPad, through context awareness, can address the typical challenges of managing low level tasks of university users. We will conduct our investigation under the assumption that all work is going on in a distributed environment where activities are attached to certain locations and thus provide support for this as well.

\section{Approach}
In order to achieve the above mentioned goals we will follow the approaches listed below

\begin{itemize}
  \item \textbf{Investigation} \newline
        We will study the research that has already been done on ABC within other domains and take the results that have relevance to a university context into consideration. We will also familiarize ourselves with the latest version of the iOS platform and learn Objective C in order to be able to implement a proof-of-concept solution.
  \item \textbf{Implement a proof-of-concept solution} \newline
        We will implement an iPad app to serve as a proof-of-concept solution. The app will serve as a client of the 6th version of the ABC framework and implement features to support context awareness, task management and mobility.
  \item \textbf{Evaluation} \newline
        We will make user tests to establish the usefulness of our proof-of-concept. We will define a set of user scenarios and have users from the target carry them out. Finally, we will record their feedback, present the results in the report in a readable format and conclude the evaluation with a reflection on whether or not it can be seen as an achievement of our goals.
\end{itemize}

\newpage

\section{Results}
We developed a proof-of-concept solution that makes students able to manage their daily activities. Based on our own experience as target domain users we defined a set of features to contribute to the overall usefulness of the solution and designed our user scenarios around those features. Next, we observed the users as they walked through the scenarios and answered surveys with questions about the experience that could be rated on a scale from 1 to 5. The evaluation clearly showed that the majority of the users were satisfied and felt that the solution would be of valuable use to them, however it also revealed that some features were clearly better than others. Equally important, most users didn't leave the experiments without suggestions to features that they thought should be improved or added. The feature that got the highest score of 4.9 (average) and a standard deviation of (0.35) was the ability to open resources in other apps whereas the overview of activities were rated with 3.9 (average) and got a standard deviation of (0.64).

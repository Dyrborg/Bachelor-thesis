\chapter{Introduction}
In this chapter, we will introduce the bachelor thesis and explain it's relevance. We will introduce the concept of Activity Based Computing and provide enough background information for the uninformed reader to understand it's purpose. At the same time, we will state an exact problem definition, explain how we approached it and briefly summarise the results. An extensive explanation of Activity Based Computing and the results of the thesis is deferred to subsequent chapters.

\section{Context and Motivation}
Today, work practices are often complex entities with respect to the number of resources and work flows involved. Take for instance a university student who follows several courses at a time. Each course might include literature to read, lectures to attend and exercises to solve. In order for a student to get through an average day, he would need to keep track of his schedule to know which course is taught that day and he would have to solve the exercises which requires resources like the document that defines them, the course web page etc. Furthermore, if he has several lectures the same day or the lecture and the exercises are held in different rooms then he would need to keep track of this and prepare a different set of documents for each activity.

As argued by \citet{bardram2011}, moving away from the typical office desk environment has a negative impact on the challenges that arise from management of parallel activities. Users are left with their own organizing skills for tools when trying to prepare and share their resources before the activity is actually happening. In modern times, computers in various shapes have been utilized to meet and optimize those challenges but they all still provide support on the lowest level only, i.e. allowing the user to browse the file system, open and close applications, manually distribute their documents to collaborators by e-mail etc.

Activity Based Computing (ABC) is a new higher level paradigm originally developed for use within hospitals. It aims at aiding users in managing parallel activities at a higher level by introducing the notion of an "Activity" that acts as the basic computational unit and facilitates easy distribution, suspension and resumption.

In this thesis we would like to investigate how the Apple iPad can be used as an ABC enabled device to aid university students in managing their daily activities. Earlier research within ABC, \citet{bardram2009} has concluded that tablets were less useful as ABC devices in hospital settings. We find investigations of the usefulness of the iPad in university settings interesting for two reasons. First, university settings resemble hospital settings by being collaborative, non-office like and attached to multiple locations which imposes amplified activity management challenges as argued by Bardram. Second, the last couple of years, tablets have undergone an evolution which improves some of their properties that have earlier been subject to criticism in Bardram's research.

\section{Background}
The ABC project started in 2002 and took it's outset in pervasive computing designed for mobile, collaborative and time critical work for clinicians in hospitals. Since 2002 a lot of experiments have been carried out with different families of devices as well as technical implementations at different software levels. The current implementation is a java peer-to-peer system that is based on the AEXO infrastructure and targeted for non-PC devices.

Devices that have been used in ABC includes desktops, laptops, tablets and wall screen displays. Tablets have received a lot of criticism due to their size and weight which made them difficult to handle in front of patients. \citet{bardram2009} even explains how nurses were forced to drop the device in the patient's lap in order to interact properly with it.

Despite the fact that ABC was designed for use in hospitals it is introduced as a paradigm for ubiquitous computing that has fostered 6 general principles (which we will elaborate on in chapter 2) and these can be applied in any context. Lately, experiments with ABC has been carried out within the field of biology.

The ABC research team is financially funded by the Danish Council for Strategic Research. The team is located at the IT University of Copenhagen and led by Jakob Bardram.

\section{Goals}
The goal is to establish the ABC related usefulness of a modern tablet computer such as the Apple iPad in a university context. We will limit our investigation to include the following ABC principles

\begin{itemize}
  \item Activity-Centered
  \item Activity Roaming
  \item Activity Adaption
  \item Activity Awareness
\end{itemize}

\section{Approach}
In order to achieve the above mentioned goals we will follow the approaches listed below.

\begin{itemize}
  \item \textbf{Investigation} \newline
        We will study the research that has already been done on ABC within other domains and take the results that have relevance to a university context into consideration when defining the features of the app. We will also familiarize ourselves with the latest version of the iOS platform and learn Objective C.
  \item \textbf{Implement a proof-of-concept solution} \newline
        We will implement an iPad app to serve as a proof-of-concept solution. The app will serve as a client of the ABC framework developed by the ITU research team and implement the 4 ABC principles mentioned in the section above. The argument for this choice of principles is deferred to a later chapter.
  \item \textbf{Evaluation} \newline
        We will make user tests to establish the usefulness of our product and argue for the feasibility of our goals. We will define a set of user scenarios and have genuine users from the target domain carry them out. Next, we will record their feedback, share some thoughts of our own and present the results in the report in a readable format as a collective evaluation.
\end{itemize}

\newpage

\section{Results}
We developed an iPad app that supports the 4 principles we decided to focus on. We defined a set of features that we considered likely to contribute to the overall usefulness of the product and designed our user scenarios around those features. Next, we had users go through the scenarios and answer surveys with questions about the experience that could be rated on a scale from 1 to 5.
The evaluation clearly showed that the majority of the users were happy with the product and felt that it would be of valuable use to them in their everyday work as students. The surveys also brought results to show that some features were clearly better than others. Equally important, most users didn't leave the room without suggestions to things that they thought should be improved or added to the solution. The feature that got the highest score of 4.9 (average) was the ability to open resources in other apps whereas the overview of activities were rated with 3.9 (average).

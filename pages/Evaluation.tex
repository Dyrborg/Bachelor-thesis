\chapter{Evaluation}
\label{sec:evaluation}
In this chapter we will test and evaluate our proof of concept. It will be tested and evaluated with regards to the goals defined in chapter 1; task-management, mobility and context awareness. The tests are conducted in order to answer the question: \emph{Is it possible to properly support the four ABC principles: Activity Centered, Activity Roaming, Activity Awareness and Activity Adaption on the iPad}.

We will evaluate the system based on thought scenarios. We want the system to be able to perform and be useful to people that are normally working at a university, such as students. We will therefore carry out functionality tests, in order to determine the usefulness of the core features in our proof of concept.

In the end we will discuss the results of the tests, and what improvement or changes should be considered for future work.

\section{Evaluation method}
\label{sec:evalutationPOC}

\citet{ugur2001} defines six steps as a guideline in order to properly carry out Human-Computer-Interaction tests:

\begin{itemize}
  \item Set the goals - \emph{What do you want to capture?}
  \item Decide on the target population and sample size - \emph{Who will you ask?}
  \item Determine the questions - \emph{What will you ask?}
  \item Pre-test the survey - \emph{Test the questions}
  \item Conduct the survey - \emph{Ask the questions}
  \item Analyze the data collected - \emph{Produce the report}
\end{itemize}

We will use these guidelines as a basis for properly defining and carry out these tests. \citet{ugur2001} further describes surveys as either quantitative or qualitative. Through quantitative surveys it is possible to get statistical data, but is not very qualitative, that is it is impossible to know \emph{why} a user likes or dislikes something in particular. Qualitative surveys are better for getting elaborated answers, but is not as useful for statistical data. In this evaluation we will use a combination of a quantitative survey with a face to face interview \citep{ugur2001}, where the interview will be based on the answers given in the survey. 

\subsubsection{Setting the goals and defining target group}
The main goal of this thesis is to find out how we can support the ABC principles; Activity Centered, Activity Awareness, Activity Adaption and Activity Roaming on the iPad. However, Activity Roaming will not be covered in this evaluation, since the goal of this thesis was to store activities in the ABC version 6 infrastructure, and not how we could roam such activities between devices.

Since the proof of concept have been developed with a university environment in mind, it would be most suitable to bring in test users that are normally working at a university. We further narrow the test users down to be students.

\subsection{Determining the questions}
It is possible to divide the questions into three categories: task-management, filtering and integration. Furthermore the questions will be formulated as statements, using the likert scale \citep{likert}. Each of the question categories will be further elaborated in the following.

\subsubsection{Task-Management}
The core concept of the proof of concept implementation is that we are able to support the notion of activities. An important functionality of the implementation is this that it should be possible to easility create these activities, but it is equally important how these are presented to the user in order for users to fully utilize activities. Another core concept of activities is the aggregation of resources. Resources, like activities, also have to be presented properly to the user, such that they are easilily accessible. 

We therefore define these questions:

\begin{description}

  \item[It is easy to create an activity.] We want to find out if the proof of concept easily support the creation of activities. 
  
  \item[The system gives a good overview of created activities.] We want to find out if the proof of concept visualize activities in a logical and usable way. 
  
  \item[I like the use of categories.] We want to determine if the use of categories makes it easier to manage and handle activities.
  
  \item[I like the use of color coding.] We want to determine if the use of color coding makes certain categories and certain activities easier to see.
  
  \item[The system gives a good overview of resources for a given activity.] Like activities we want to know if resources are presented and visualized in a logical and usable way, but also to find out if these two concept should be handled differently, instead of similarly.
  
  \item[The ability to easily save a website that you are visiting is useful.] We wish to know if users want to be able to quickly add a website they are visiting, instead of just writing the URL directly into a dialog box.
  
  \item[I had all features available in order to easily complete the tasks.] This question might seem like a more usability minded question, but the intend is to force the user to think about if some core functionality is missing, in order for them to better handle the scenario that they are asked to complete.
  
\end{description}

\subsubsection{Filtering}
A very important part of the proof of concept is the use of location filtering, and it is therefore very important to get user feedback on how this works, and how useful they think it is. Furthermore we implemented the notion of categories, and also a filtering option based on this. An interesting result is also to see which of these filtering methods is perceived as the most useful.

We therefore define these questions:

\begin{description}
  \item[I find the category filtering useful.] We want to get feedback whether or not this kind of filtering is perceived as useful. 
  
  \item[I find the location filtering useful.] We want to get feedback whether or not this kind of filtering is perceived as useful. This is particularly important since the ABC paradigm defines Activity Awareness as a core concept, where activities are able to adapt based on its environment, and the result of this question could determine if this is a valid concept to use on the iPad.
\end{description}

\subsubsection{Integration}
Integration was done based on the paper that involved integrating ABC into a desktop environment. As discussed in chapter 3, it was important to integrate the proof of concept as much as possible into the existing operating system, but also by using cloud services, and provide an interface that worked as the basic interaction with the device as a whole. We therefore want to find out how desired and useful the implemented solutions are.

We therefore define these questions:

\begin{description}

  \item[The ability to add resources from dropbox, image library and the camera is useful.] Since we integrated three external systems that handles files, from which resources could be added, it is interesting to find out if this is a good solution for retrieving resources.
  
  \item[I find the all-time access to the browser useful.] It was decided that the a browser should make up most of the UI space, and it would be interesting to find out if this a desired solution, or if it should be hidden until needed.
  
  \item[I find the integration with native apps useful.] Last but not least, it is interesting to determine if integration with local apps is a desired and usable feature.
  
\end{description}

\section{The Setup}
As explained earlier, we wanted the users to participate by doing thought scenarios that takes place at a real university. The scenarios should be constructed such that they support the questions defined in section \ref{sec:evalutationPOC}.

We came up with two scenarios: one that focuses on creating and managing activities and resources, as well as some of the integration solutions, and one that focuses on filtering and making use of local applications. The test is conducted at the IT-University of Copenhagen.

The two scenarios are as follow:
\par\vspace{\baselineskip}

\textbf{Scenario 1}
\begin{quotation}
\emph{
A student have a busy day tomorrow at the University. In the morning the student have to do a presentation in an Auditorium, which involves a PDF presentation, document notes and some sample websites. After the presentation the student have a meeting with his supervisor regarding a school project he is doing at the supervisors office. They will discuss several designs of a product the student is developing, and involves meeting notes, some online resources and several images of the design. Later that day the student need to attend a lecture on an interesting subject which requires him to have access to the lecture slides, the exercises presented, and a note document. This will take place in a small teaching room. At the end of the day the student will be attending workshop on innovation. This will include a website, and a sketch of a brilliant idea that the student would like som feedback on. The location for this activity have not been dertermined yet.
}
\end{quotation}

\textbf{Scenario 2}
\begin{quotation}
\emph{
The day have come to carry out yesterdays preparations. The student will begin by going to 2C to do his presentation using the resources prepared yesterday. In the break during his presentation he wants to look through the rest of the presentation on in the local iBooks application. He finishes up his presentation and proceed to the meeting with his supervisor in 4C. He have a very fruitfull meeting, and are able to present all of his designs and ideas. They had a long discussion and a lot of drawings on the whiteboard, and the student desides to take a picture and add it to the activity. He also wants to add some comments to his notes, and opens up his note-PDF in the local GoodReader application. After a short break he moves on to attend his lecture in 4E. During the lecture he takes important notes, and feels refreshed by all the new things he have learned. He also finds a good tutorial online that he adds to his lecture activity. Last but not least he meets up with the other innovators, and they move around the university and find a suitable and available room. He finishes the workshop and head home.
}
\end{quotation}

Each survey will be organized as follow:
\par\vspace{\baselineskip}
\textbf{Survey - Scenario 1}
\begin{itemize}
	\item It is easy to create an activity.
	\item The system gives a good overview of created activities.
	\item The system gives a good overview of resources for a given activity.
	\item The ability to easily save a website that you are visiting is useful.
	\item The ability to add resources from dropbox, image library and the camera is useful.
	\item I find the all-time access to the browser useful.
	\item I had all features available in order to easily complete the tasks.
\end{itemize}

\textbf{Survey - Scenario 2}
\begin{itemize}
	\item I like the use of categories.
	\item I like the use of color coding.
	\item I find the category filtering useful.
	\item I find the location filtering useful.
	\item I find the integration with native apps useful.
\end{itemize}

Initially it was necessary to create a known environment of activities in the application, from which the test user could familiarize themselves. It is also necessary to have more than the four activities that the test users are required to create during the first scenario. A student at a university would probably have at least two or three times that amount, to account for lectures, meetings, exercises and so on. It would also help to further emphazise if location and category filtering would be used, and the usefulness of color coding and category assignment for an activity. The example activities that are created prior to each test is shown in figure \ref{fig:initial}.

\begin{figure}[ht!]
  \centering
    \includegraphics[scale=0.35]{startStateTest}
  \caption{\emph{The inital setup for a user. To the left are shown the example activities, with their category and color coding attached. To the right is the initial state of the browser.}}
  \label{fig:initial}
\end{figure}

Before each test, the user were given a brief introduction to the concept of activities and its resources. This was followed by a brief introduction to what functionalities the proof-of-concept offered to the users, and afterwards the user were allowed to use a a couple of minutes to get aquinted with the proof-of-concept. The users were then asked to complete each scenario seperately, and after each scenario they were given the survey for that scenario. When scenario 2 and the scenario 2 survey was completed, the user participated in an interview based on the answers given in both surveys.

\section{Results}
8 users were recruited for the tests. The profile for each user will be presented in the following:

\begin{description}

\item[Test user 1] \hfill \\
	Male, 26 years old, iPad user, Software Development student and advanced Mac user.
\item[Test user 2] \hfill \\
	Male, 22 years old, iPad user, Software Development student and advanced PC user.
\item[Test user 3] \hfill \\
	Female, 24 years old, not an iPad user, familiar with iOS on the iPhone, E-Bussiness student and advanced PC user.
\item[Test user 4] \hfill \\
	Female, 23 years old, not an iPad user, unfamiliar with iOS on any device, BioTechnology student and standard PC user.
\item[Test user 5] \hfill \\
	Male, 21 years old, not an iPad user, familiar with iOS on the iPhone, Software Development Student and advanced PC user.
\item[Test user 6] \hfill \\
	Female, 22 years old, not an iPad user, familiar with iOS on the iPhone, Organizational Learning student and standard Mac user.
\item[Test user 7] \hfill \\
	Male, 21 years old, not an iPad user, unfamiliar with iOS on any device, Software Development student and advanced PC user.
\item[Test user 8] \hfill \\
	Male, 22 years old, iPad user, Software Development student and advanced Mac user.
\end{description}

Each user completed both scenarios and both surveys, and each user participated in the interview that followed.

\subsubsection{Quantitative Results}

\begin{table}[ht]
\begin{center}
    \begin{tabular}{ | p{7cm} | c | c |}
    \hline
    \textbf{Question} & \textbf{Avg score} & \textbf{Std. dev.}\\ \hline
    		(1) It is easy to create an activity. & 4.4 & 0.74 \\ \hline 
		(2) The system gives a good overview of created activities. & 3.9 & 0.64 \\ \hline
		(3) The system gives a good overview of resources for a given activity. & 4.5 & 0.76 \\ \hline
		(4) The ability to easily save a website that you are visiting is useful. & 4.8 & 0.46 \\ \hline
		(5) The ability to add resources from Dropbox, image library and the camera is useful. & 4.9 & 0.35 \\ \hline
		(6) I find the all-time access to the browser useful. & 4.1 & 0.99\\ \hline
		(7) I had all features available in order to easily complete the tasks. & 3.9 & 0.99 \\ \hline
		(8) I like the use of categories. & 4.0 & 0.53 \\ \hline
		(9) I like the use of color coding. & 4.1 & 0.83\\ \hline
		(10) I find the category filtering useful. & 4.4 & 0.74\\ \hline
		(11) I find the location filtering useful. & 4.4 & 1,06\\ \hline
		(12) I find the integration with native apps useful. & 4.9 & 0,35\\ \hline
	\end{tabular}
\end{center}
\caption{\emph{The avg. result for each of the survey questions, based on the answer of 8 users. 1 is the lowest possible score and 5 is the highest possible score}}
\label{table:quantitativeResult}
\end{table}

In generel most of the features are found useful by the users. As can be seen in table \ref{table:quantitativeResult} , the test users found especially three features useful; Integration with native applications, the ability to add resources from Dropbox, image library and camera, and the ability to easily save a website that you are using. The features that scored the least were the \emph{overview of activities} and if the user \emph{had all features available in order to easily complete the tasks}. Based on these results it is especially interesting that the overview of activities scored lower than the overview of resources. One of the core functionalities, location tracking scored rather high, but not as high as some other features. Following these results it is now interesting to consider the feedback from the users, on why they scored the different features the way they did.

\section{Discussion}
In the introduction of this thesis we defined three goals that we wanted to support; task management, mobility and context awareness. Based on the results in table \ref{table:quantitativeResult} we will discuss if the goals have been reached. During the discussion the results from table \ref{table:quantitativeResult} will be supported by quotes from the interview with each of the test users. 

\subsubsection{Task Management}
After the test had been conducted with each of the test users, one important thing came to our attention; we had not asked the test users whether they found the use of activities useful. This is a major problem, as this basically makes it hard to determine if the concept of activity based computing was a success to implement on the iPad. However, the question \emph{I had all the features available to easility complete the tasks}, did provide us with some valuable feedback on the use of this concept during the interviews. Furthermore this proved the importance of doing the interviews, as it was possible to get feedback that was not necessarily covered by the questions.

To begin with we consider three statements that were noted during the feedback.

\begin{quotation}
	- I 	like that you are able to associate resources with an activity. Seems way more logical, instead of having to keep track of your files in a file system.
\end{quotation}

\begin{quotation}
	- The use of activities is somewhat similar to how I try organize my files on my Mac. This just makes it easier to handle, since I dont have to navigate through alot of folders first.
\end{quotation}

\begin{quotation}
	- If it was possible associate activities with certain projects or topics as well, then this organization of files would be really helpful.
\end{quotation}

Each of these statements shows that, for at least three users, the use of activities to aggregate resources were much more suitable than what they were used to. One of them even stated that he tried to achieve the same thing on his Mac, by organizing folders and files into "activity" folders, but the use of activities made it more easy to handle. Based on this feedback one can conclude that the principle of Activity Centered was a success, but only by keeping in mind that this was only based on the feedback from three of the eight users - and not by all the test users. Furthermore, since this was not a specific question, it is not possible to root out that some of the test users disliked the use of activities for task management. In conclusion one can say that the use of activities were perceived helpful to some of the users and that the goal were achieved for these users.

Another aspect of task management, is how one could visually represent activities to the user. This question was however, covered by the survey by, \emph{The system gives a good overview of created activities} and scored 3.9 (std. dev. 0.64). This tells us that users only slighty agreed that this overview was good. The feedback on this is interesting because the things that made it good for some users, made it poor for others:

\begin{quotation}
	\emph{
		- I really like the use of color coding and categories in the overview - it makes the overview better in an unsorted list [...]
	}
\end{quotation}

\begin{quotation}
	\emph{
		- Colors and categories are very nice, and gives and easy overview of what kind of activities you have to do.
	}
\end{quotation}

\begin{quotation}
	\emph{
		- Activies are hard to get an overview of, because there are too many unfamiliar colors and categories.
	}
\end{quotation}

\begin{quotation}
	\emph{
		- The overview is confusing. Would be nice with some sort of sorting, or if it would be possible to define the colors and categories yourself. I really likes the concept of color coding and categories though.
	}
\end{quotation}

This feedback tells us that the use of categories and colors were only useful to some, and could be more useful for others if they could define the categories and colors themselves. The concept of categories and colors were implemented as an inspiration from how trains in Copenhagen, Denmark used colors and letters to identify specific trains, however this solution seemed to be misplaced in this context. One could solve this by making it an optional entity - such that users that would find it useful could use these identifiers, and users that find them confusing, could avoid using them at all. However for resources the use of categories and familiar images for specific types of files received a better score of 4.5 (3), and based on the feedback it was mainly due to two things, familiarity and known categories:

\begin{quotation}
	\emph{
		- The resource overview works better than the activity overview, since there are fewer categories, and the images are well known.
	}
\end{quotation}

\begin{quotation}
	\emph{
		- In the resource overview, resources are easier to identify, because the categories and its associated image is more well known.
	}
\end{quotation}
	
These two statements from two different users are quite similar, which further emphasized that the concept of categories itself as well as using a visual identifier (in this case an image) is not that bad. However the concept might only be fit for resources, where users are more used to have an image for different types of files, and where the use of categories and file types is much more similar to what users experience in a desktop environment. In conclusion, the use of categories and colors/images as an identifier for activities is not a generelly feasible solution, however it is a feasible solution for resources, where this concept is more similar to what is experienced on a desktop computer.

The last important part of supporting tasks, is that users are able to make use of existing, known services and applications. In the survey two questions handled this issue (5)(12), and both got the highest scores of any question, and had the lowest standard deviations as well, concluding that these features were something that the users to a high degree found usable. The integration with existing services and programs were split into two parts: local files such as images and Dropbox and local program. The integration with images and Dropbox, was popular highly due to the fact that these were something that the users were using normally:

	\begin{quotation}
		\emph{
			- Integration wit dropbox is very nice. I rarely use any applications on the iPad that does not have Dropbox support, since i got all my work related files there.
		}
	\end{quotation}

	\begin{quotation}
		\emph{
			- We often conduct biological experiments, which we document by taking pictures of the setup, so the integration with the camera is a very useful feature, since it make it possible to add it directly to your activity.
		}
	\end{quotation}
	
	\begin{quotation}
		\emph{
			- In design projects one usually use a lot of images from your computer or camera, so being able to add images from the local image library is a very nice feature.
		}
	\end{quotation}
	
	This proves that to best support task management, it is important that the users are allowed to use features that they normally rely upon, in order to carry out activities. When it comes to the integration with local applications the argument is similar:
	
	\begin{quotation}
		\emph{
			- I really like the integration with the local applications. I simply hate when I am not able to use my favorite programs for what I need to do.
		}
	\end{quotation}
	
	\begin{quotation}
		\emph{-	Integration with the native applications is very nice, since it makes it possible to handle resources in your favorite program, and not being forced to use a specific pre-defined one.}
	\end{quotation}
	
	\begin{quotation}
		\emph{-	Integration with native apps is brilliant.}
	\end{quotation}

	This feedback proves that a generel integration is vital for task-management to be fully supported, which is also why it was implemented in the proof of concept. It could have a great impact if users had to relearn new services and applications in order to carry out their normal tasks, instead of using what they were used to. In conclusion, in order to support task-management, it is very important that activities integrate with existing and familiar services and applications, which the user can freely make use of.

\subsubsection{Mobility}
The goal of mobility were not tested, as the implementation focused on storing activities in the ABC version 6 infrastructure. To test this would require users to access the same activities on different devices, and we simply did not have more devices available. However, the proof of concept does make use of the version 6 infrastructure to store activities, meaning that they are not stored locally on the device, and can be accessed from anywhere. This way the ABC principle, Activity Roaming, which aims at storing activities in an infrastructure that can be roamed across devices, have been achieved. It is also possible to have a similar proof of concept implementation running on another iPad, and retrieve all created activities. 

However limitations on the version 6 infrastructure made a couple of things impossible to store in the infrastructure. The introduction of colors and categories were not supported in the infrastratucture, and thus it was necessary to save the mapping between activities and their associated categories and colors locally. This meant that if one were to reset all data on the client, and retrieve all activities anew from the infrastruture, none of them would have a category or color. If the infrastructure were to support these, then this information would be retrieved whenever an activity was fetched from the infrastructure and would be displayed - no matter what iPad were running our proof of concept.

The same limitation existed when attaching locations to activities. Since the infrastructure does not offer a way to give an activity an associated location, this information had to be stored locally as well. This also meant that were the activities retrieved on another iPad, this information would be lost, and it would be impossible to use location tracking. Were this information supported by the infrastructure, this information could of course be stored with the activity, and make it usable across devices.

It was suggested by ABC research team member Steven Houben, that as a solution one could make use of the list of metadata entities that is stored with an activity. However this would mean that all information about location, categories, colors and so on would be saved in a pure text format - which would require all clients to agree on a way to format such information, and parse this text in a uniform way once an activity were retrieved. It was decided not to use such a solution, but instead suggest that activities were able to store a location, since this concept were used in this project, is an important part of the ABC paradigm \citep{bardram2011}, and have been used in other ABC related work \citep{bardram2012}, \citep{bardram2009}. It was not suggested that activities stored categories and colors as this was a concept we introduced, and that such a suggestion should depend on the result of our evaluation.


In conclusion we support mobility in the sense that activities are stored and roamed through an infrastructure, however information on categories, colors and locations are stored locally due to infrastructure limitation.

\subsubsection{Context Awareness}
The goal of context awareness were based of the location filtering feature. The feature scored 4.4 in the survey (11), but had the highest standard deviation of all the questions. Looking at the individual answers, this is due because one of the users scored this feature a lot lower than the rest, thus creating a much larger diversion from the average answer.
To begin with we start by considering some of the positive feedback:

\begin{quotation}
	\emph{
		- I really like that the system suggest what I can do in the room that I am in right now. Then I don't need to think about it.
	}
\end{quotation}

\begin{quotation}
	\emph{
		- The location filtering really help to improve the lack of sorting. Suddenly you are only shown a couple of activities instead of a whole list.
	}
\end{quotation}

\begin{quotation}
	\emph{
		- Location tracking in a university looks like it have great potential based on this solution.
	}
\end{quotation}

The feedback shows that the feature is a very powerful one. A couple of the users, like the first one here, says that the feature is great at helping them think less, when using the proof of concept. This is a very strong argument as \citet{ugur2001} argues that it is important to remember that users have less cognitive function when using mobile devices. If the users thus feel, that this helps them in this particular way, then it can be considered a great success. It is also stated that the feature helps when there are no sorting options available. However one can argue that such options should be available, which were also argued by all the test users. However one did \emph{disagree} that this feature was useful:

\begin{quotation}
	\emph{
		- Location filtering only seem useful when you know where to go. I rarely know where my classes takes place as it changes alot, and are often notified just before the lecture takes place.	
	}
\end{quotation}

The entire concept of context awareness, requires not only that a client, like our proof of concept, are able to find out about its current context, but also that this information can be mapped to an activity. As this user describes, it is rarely the case that such information is available, which does render a feature such as location filtering rather useless. The same user however scored category filtering much higher, because it was a much better way of filtering activities when considering the location problem.
This concludes two things: first, location tracking is a useful feature, if and only if, it is possible to know where activities takes place. Second, when such information is \emph{not} available there should be other filtering options available.

\section{Suggestions for future improvements}
The survey question \emph{I had all features available in order to easily complete the tasks}, made it possible for the users during the interview to speak freely of what they thought would improve the use of the system. In the following we will therefore present some of the most notable, as well as most mentioned improvement for future work.

\subsubsection{Calendar}
Several of the users stated that they really missed an option to declare a date, and a time of the day of an activity. They mentioned that when one is at a university, it is usually only relevant to see what activities will happen today, and not tomorrow. Several of the users also expressed a concern with the current solution because they did not think that they would ever delete completed activities, and over time that would eventually involve having quite a lot of activities present in the activity overview. Having activities support time scheduling, would also make it possible to have the proof of concept work like an advanced calendar. One user even suggested that all appointments in the local calendar program on the iPad would automatically be retrieved an created as activities, that you could add resoruces to.

\subsubsection{Metadata}
In correlation with location filtering but also the above mentioned calendar feature, it should be possible to easily see this information for an activity. Many users complained that it was not possible to access this information, after an activity was completed. They were especially concerned with the fact that they might forget where an activity takes place, and then there would be no way through the proof of concept solution to find out.

\subsubsection{Editing}
It became clear very quickly that the lack of editing would have a negative impact on the user evaluation. We had not thought about this during evaluation, but we decided to complete the user test without chaning this. As was clear during the evaluation users were generally very quick to click their way throught he creation screens for both activity and resources, and often they wanted to change something they had already creation. Due to the lack of editing they were forced to delete what they had just created, and create the same thing anew witht the new changes. This caused a lot of frustration and should therefore be implemented. But another and even stronger argument was presented by one of the users:

\begin{quotation}
	\emph{
		- At my school there are a lot of problems with getting enough room for all the lectures, which result in lectures being moved around all the time, and its a big handicap if it is not possible to change the location on an activity that you have already created.
	}
\end{quotation}

The possibility to edit activities as well as resources should there be implemented in order to support such situations.

\subsubsection{Sorting}
The major point of critique of the activity overview, is that it lack sortings capabilities. Many users perceived location tracking a good solution to improve this, but that would not help if you were in a situation where you could not rely on this. Users specifically asked for the option to sort on name, time of day, date, collated categories (all meetings stand next to each other, all lectures stand next to each other and so on) or creation time (newest first). Users had the option to filter based on categories, but only found this useful if one had quite a lot of activities to filter on. In you just had a the current list full it would probably not be use as sorting would be more appropriate.